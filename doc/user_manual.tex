
%%%%%%%%%%%%%%%%%%%%%%%%%%%%%%%%%%%%%%%%%%%%%%%%%%%%%%%%%%%%%%%%%%%%%
% FOR PRINTING
\documentclass[
    11pt,
    a4paper,
    openright
]{report}
               
\newcommand{\litl}{LiTL}
\newcommand{\reporttitle}{\litl}               
\newcommand{\reportsubtitle}{Lightweight Trace Library}
\newcommand{\reportsubsubtitle}{User Manual}
%%%%%%%%%%%%%%%%%%%%%%%%%%%%%%%%%%%%%%%%%%%%%%%%%%%%%%%%%%%%%%%%%%%%%
\usepackage{etex}
\usepackage[utf8]{inputenc}
\usepackage{pdflscape}
\usepackage[T1]{fontenc}
\usepackage[small,bf]{caption}

% for plotting 
\usepackage{subfigure}
\usepackage{graphicx}
\usepackage{color}  
\usepackage{xcolor}
    \definecolor{ListingsKeywordColor}{rgb}{0,0,0.4}
    \definecolor{ListingsIdentifierColor}{rgb}{0,0.5,0}
    \definecolor{ListingsCommentColor}{rgb}{0.4,0.4,0.4}
    \definecolor{ListingsStringColor}{rgb}{0.6000,0.3333,0.7333}%{0.8,0,0}
    \definecolor{ListingsRuleSepColor}{rgb}{0,0,0}
    \definecolor{ListingsEmphColor}{rgb}{0,0.6667,0.6667}
    \definecolor{ListingsBreakSymbolColor}{rgb}{0.780,0.082,0.522}
    \definecolor{LinkColor}{rgb}{0,0,0.5}
    \definecolor{UnitColor}{rgb}{0,0,0}
    \definecolor{MathsVectorColor}{rgb}{0,0,0}
    \definecolor{MathsMatrixColor}{rgb}{0,0,0}
    \definecolor{MyGreen}{HTML}{228B22}
    \definecolor{MyBlue}{HTML}{0000FF}
    \colorlet{MatrixElementsLight}{gray!20!white}
    \colorlet{MatrixElementsDark}{gray!40}

\usepackage{tikz}
\usepackage{pgfplots}
\usetikzlibrary{
  external,
  arrows,
  positioning,
  decorations.pathmorphing,
  3d
}
\tikzexternalize
\tikzsetexternalprefix{imgs/tikz/}
\tikzset{
    external/export=false,
    %Define standard arrow tip
    >=stealth',
    %Define style for boxes
    punkt/.style={
           rectangle,
           rounded corners,
           draw=black, very thick,
           text width=6.5em,
           minimum height=2em,
           text centered},
    % Define arrow style
    pil/.style={
           ->,
           semithick,
           shorten <= 0pt,
           shorten >= 0pt,},
    pild/.style={
           ->,
           thick,
           %>=angle 90,
           shorten <= 0pt,
           shorten >= 0pt,}
}

\pgfkeys{
    /tikz/external/mode=list and make
}
\pgfplotsset{
  xtick scale label code/.code={$\times 10^{#1}$}
}
\pgfplotsset{
ytick scale label code/.code={$\times 10^{#1}$}
}
\pgfplotsset{
  invoke before crossref tikzpicture={\tikzexternaldisable},
  invoke after crossref tikzpicture={\tikzexternalenable},
}
    
% for listing
\usepackage{listings}
  \lstset{
    basicstyle=\scriptsize\ttfamily,
    tabsize=3,
    showtabs=false,
    showspaces=false,
    showstringspaces=false,
    tab=\rightarrowfill,
    keywordstyle=\color{ListingsKeywordColor},
    identifierstyle=\color{ListingsIdentifierColor},
    commentstyle=\color{ListingsCommentColor},
    stringstyle=\color{ListingsStringColor},
    emphstyle=\color{ListingsEmphColor}\bfseries\underbar,
    frame=none,
    rulesepcolor=\color{ListingsRuleSepColor},
    numbers=left,
    numberstyle=\tiny,
    numbersep=5pt,
    captionpos=top,
    frame=tb,
    firstnumber=1,
    stepnumber=1,
%     numberfirstline=false,
    breaklines=true,
    breakatwhitespace=true,
%     prebreak=\mbox{\,$\color{ListingsBreakSymbolColor}\mathbf{\hookleftarrow}$},
    mathescape=true,
    morekeywords={},
}

\usepackage{url}
\usepackage[final]{hyperref}
\hypersetup{
   % Farben fuer die Links
   colorlinks=false,         % Links erhalten Farben statt Kaeten
   urlcolor=LinkColor,    % \href{...}{...} external (URL)
   filecolor=LinkColor,  % \href{...} local file
   linkcolor=LinkColor,  %\ref{...} and \pageref{...}
   menucolor=LinkColor,
   citecolor=LinkColor,
   % Links
   raiselinks=true,       % calculate real height of the link
   breaklinks,              % Links berstehen Zeilenumbruch
   %backref=page,            % Backlinks im Literaturverzeichnis (section, slide, page, none)
   %pagebackref=true,        % Backlinks im Literaturverzeichnis mit Seitenangabe
   verbose,
   % hyperindex=true,         % backlinkex index
   linktocpage=true,        % Inhaltsverzeichnis verlinkt Seiten
   % hyperfootnotes=false,     % Keine Links auf Fussnoten
   % Bookmarks
   % bookmarks=true,          % Erzeugung von Bookmarks fuer PDF-Viewer
   bookmarksopenlevel=1,    % Gliederungstiefe der Bookmarks
   bookmarksopen=false,      % Expandierte Untermenues in Bookmarks
   bookmarksnumbered=true,  % Nummerierung der Bookmarks
   bookmarkstype=toc,       % Art der Verzeichnisses
   % Anchors
   plainpages=false,        % Anchors even on plain pages ?
   pageanchor=true,         % Pages are linkable
   % PDF Informationen
   pdftitle={\reporttitle: \reportsubtitle. \reportsubsubtitle},             % Titel
   pdfauthor={Roman Iakymchuk},            % Autor
   pdfcreator={LaTeX, hyperref, KOMA-Script},
   pdfstartview=Fit,       % Dokument wird Fit Height geaefnet
   pdffitwindow=true,
   pdfpagemode=UseOutlines, % Bookmarks im Viewer anzeigen
   % pdfpagelabels=true,      % set PDF page labels
}

\usepackage{cleveref}
\Crefname{figure}{Fig.}{Figs.}

\newcommand{\bytes}{b}
\newcommand{\kb}{Kb}
\newcommand{\mb}{Mb}
\newcommand{\ghz}{GHz}
\newcommand{\flop}{Flop}
\newcommand{\flops}{Flops}
\newcommand{\gflops}{G\flops}
\newcommand{\eztrace}{EZTrace}
\newcommand{\fxt}{FxT}
\newcommand{\pthread}{Pthreads}
\newcommand{\openmp}{OpenMP}
\newcommand{\mpi}{MPI}
\newcommand{\dash}{ -- }


% Title Page
\title{
{\Huge\bf \reporttitle{}}\\[6mm]
{\LARGE\bf \reportsubtitle}\\[12mm]
{\Large\bf \reportsubsubtitle}}
\author{Roman Iakymchuk and Fran\c{c}ois Trahay}

\begin{document}
\maketitle

\tableofcontents

\chapter{License of \litl}
\litl{} is developed and distributed under the GNU General Public License.

This library is free software; you can redistribute it and/or modify it under 
the terms of the GNU General Public License as published by the Free Software 
Foundation; either version 2 of the License, or (at your option) any later 
version.

\litl{} is distributed in the hope that it will be useful, but WITHOUT ANY 
WARRANTY; without even the implied warranty of MERCHANTABILITY or FITNESS FOR A 
PARTICULAR PURPOSE. See the GNU General Public License for more details.

You should have received a copy of the GNU General Public License along with 
\litl{}; if not, write to the Free Software Foundation, Inc., 675 Mass Ave, 
Cambridge, MA 02139, USA.


\chapter{\litl}
\litl{}~\cite{litl} is a lightweight binary trace library that aims at 
providing performance analysis tools with a scalable event recording mechanism 
that utilizes minimum resources of the CPU and memory. In order to efficiently 
analyze modern HPC applications that combine \openmp{} (or \pthread) threads 
and \mpi{} processes, we design and implement various mechanisms to ensure the 
scalability of \litl{} for a large  number of both threads and processes. 

\litl{} is designed in order to resolve the following performance tracing 
issues:
\begin{itemize}
 \item Scalability and the number of threads;
 \item Scalability and the number of recorded traces;
 \item Optimization in the storage capacity usage.
\end{itemize}
As a results, \litl{} provides similar functionality to standard event recording 
libraries and records events only from user-space. \litl{} minimizes the usage 
of the CPU time and memory space in order to avoid disturbing the application 
that is being analyzed. Also, \litl{} is fully thread-safe that allows to record 
events from multi-threaded applications. Finally, \litl{} is a generic library 
that can be used in conjunction with many performance analysis tools. 


\chapter{Installation}
\section{Requirements}
In order to use \litl{}, the following software is needed:
\begin{enumerate}
 \item autoconf 2.63;
 
 \item libelf or libbfd. On Debian, libelf can be installed from command line by 
       the following command:\\
       \texttt{apt-get install libelf-dev}
\end{enumerate}

\section{Getting \litl}
Current development version of \litl{} is available via Git\\
    \hspace*{0.9cm}\texttt{git clone git+ssh://fusionforge.int-evry.fr//var/lib/}\\
    \hspace*{0.9cm}\texttt{gforge/chroot/scmrepos/git/litl/litl.git}\\
After getting the latest development version from Git,\\
\hspace*{0.9cm}\texttt{./bootstrap}\\ 
should be run in the root directory and only then the tool can be built.

\section{Building \eztrace{}}
At first, to configure \litl{} the following configure script should be 
invoked:\\
    \hspace*{0.9cm}\texttt{./configure --prefix=<LITL\_INSTALL\_DIR>}\\
The configuration script contains many different options that can be set. 
However, during the first try we recommend to use the default settings.

Once \litl{} is configured, the next two commands should be executed:\\
    \hspace*{0.9cm}\texttt{make}\\
    \hspace*{0.9cm}\texttt{make install}

In order to check whether \litl{} was installed correctly, a set of tests can be 
run as\\
    \hspace*{0.9cm}\texttt{make check}

\chapter{\litl{} in Details}
    
\chapter{How to Use \litl{}}
\section{Reading Events}
After the application was traced and events were recorded into binary trace 
files, those traces can be analyzed using \texttt{litl\_read} as\\
    \hspace*{0.9cm}\texttt{litl\_read -f trace.file}\\
This utility shows the recorded events in the following format:
\begin{itemize}
 \item Time since last probe record on the same CPU;
 \item ID of the current thread on this CPU;
 \item Event type;
 \item Code of the probe;
 \item Number of parameters of the probe;
 \item List of parameters of the probe, if any.
\end{itemize}

\section{Merging Traces}
Once the traces were recorded, they can be merged into an archive of traces for
further processing by the following command\\
    \hspace*{0.9cm}\texttt{litl\_read  -o archive.trace trace.0 trace.1 ... trace.n}

\section{Splitting Traces}
If there is a need for a detail analysis of a particular trace files, an archive
of traces can be split back into separate trace files by\\
\hspace*{0.9cm}\texttt{litl\_read  -f archive.trace -d output.dir}


\chapter{Environment Variables}
For a more flexible and comfortable usage of \litl{}, we provide three 
environment variables:
\begin{itemize}
 \item \texttt{LITL\_BUFFER\_FLUSH} specifies the behavior of \litl{} when the 
       event buffer is full. If it is set to one, which is a default option, 
       the buffer is flushed. This permits to record traces that are larger 
       than the buffer size. Otherwise, if it is set to zero any additional 
       event will be recorded. The trace is, thus, truncated and there is no 
       impact on the application performance;

 \item \texttt{LITL\_THREAD\_SAFETY} specifies the behavior of \litl{} while
       tracing multi-threaded applications. If it is set to one, which is a 
       default value, the thread safety is enabled. Otherwise, when it is set 
       to zero, the event recording is not thread safe;
 
 \item \texttt{LITL\_TIMING\_METHOD} specifies the timing method that will be 
       used during the recording phase. The \litl{} timing methods can be 
       divided into two groups: those that measure time in clock ticks and 
       those that rely on \texttt{clock\_gettime()} function. The first has one 
       method:
       \begin{itemize}
        \item \texttt{ticks} that uses the CPU specific register, e.g. rdtsc 
        on X86 and X86\_64 architectures.
       \end{itemize}     
       The second has five different possibilities:
       \begin{itemize}
        \item \texttt{monotonic} that corresponds to \texttt{CLOCK\_MONOTONIC};
        \item \texttt{monotonic\_raw}\dash{}\texttt{CLOCK\_MONOTONIC\_RAW};
        \item \texttt{realtime}\dash{}\texttt{CLOCK\_REALTIME};
        \item \texttt{thread\_cputime}\dash{}\texttt{CLOCK\_THREAD\_CPUTIME\_ID};        
        \item \texttt{process\_cputime}\dash{}\texttt{CLOCK\_PROCESS\_CPUTIME\_ID}.
       \end{itemize}
       User can also define its own timing method and set the environment 
       variable accordingly.
\end{itemize}


\chapter{Troubleshooting}
If you encounter a bug or want some explanation about \litl{}, please contact 
and ask our development team on the development mailing list
\begin{itemize}
 \item \url{litl-devel@fusionforge.int-evry.fr}.
\end{itemize}


\bibliographystyle{plain}
\cleardoublepage
\phantomsection
\addcontentsline{toc}{chapter}{Bibliography}
\small
\bibliography{references}
\normalsize

\end{document}
